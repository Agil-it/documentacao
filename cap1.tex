\chapter{Introdução ao Documento}

{A faculdade SENAI (Serviço Nacional de Aprendizagem Industrial) \footnote{\url{http://sc.senai.br/pt-br/faculdade-senai-jaragua-do-sul}}  visa sistematizar os conhecimentos adquiridos pelos estudantes durante o decorrer do curso, como também, oferecer vivência  prática-profissional mediante a aplicação dos conhecimentos em situações reais. Além disso, busca  propiciar ao estudante o contato com o universo acadêmico da iniciação científica, através da implantação do \textit{Projeto Integrador}, que tem exatamente essas características em locais que está sendo implantado. Portanto o Senai está em parceria com a empresa Duas Rodas desenvolvendo este projeto. }



A empresa Duas Rodas possui sua sede em Jaraguá do Sul, Santa Catarina e tem como principal ramo a indústria de alimentos \footnote{\url{ https://www.duasrodas.com/}}. Para a execução de suas tarefas diárias, seus funcionários contam com o auxílio direto de máquinas de pequeno, médio e grande porte, e sabe-se que esses equipamentos necessitam, de manutenção preditiva e corretiva  de tempos em tempos. Para tanto, se faz necessário ajustes adequados e rápidos para que não haja baixa produtividade e consequentemente uma redução em seus proventos.

Esses equipamentos precisam passar por uma série de cuidados e estar em bom estado para o uso dos seus colaboradores. Portanto, é de extrema importância para empresa Duas Rodas ter um controle e um bom fluxo de manutenções desses equipamentos, de modo a não haver uma inatividade desnecessária, e quando necessário a manutenção, esta seja rápida e eficiente.

% CAP 1.1
% ---
\section{Tema}
% ---
O projeto consiste no desenvolvimento de um sistema de ordem de manutenção de máquinas e equipamentos que busca minimizar os trâmites burocráticos e demorados que hoje está sendo utilizado.

% CAP 1.2
% ---
\section{Objetivo do Projeto}
% ---
O objetivo do projeto é que a empresa alimentícia Duas Rodas consiga digitalizar
o fluxo de manutenção de seus equipamentos não afetando a demanda de serviço e facilitando a execução das OMs que encontra-se em fila, reduzindo assim a utilização excessiva de papel.

% CAP 1.3
%---
\section{Delimitação do Problema}
%---
Atualmente a empresa Duas Rodas gera ordens de manutenção no SAP e imprime um formulário padrão. Esse formulário é passado para o manutentor, que realizará a manutenção e preencherá todos os campos necessários do formulário. Após a finalização da manutenção, os usuários responsáveis assinam esse formulário para comprovar que o problema foi resolvido. Após as assinaturas, o formulário é enviado a um usuário administrativo que irá aprovar e enviar ao SAP.
% ---

% CAP 1.4
% ---
\section{Justificativa da Escolha do Tema}
% ---
Este projeto surgiu por meio da deficiência encontrada nos setores de manutenção da empresa, onde os mesmos relataram os problemas e dificuldades na execução de manutenção nas máquinas e equipamentos. Desta forma, em parceria com a instituição de ensino SENAI e o Curso de Sistemas para Internet de Jaraguá do Sul, resolveu-se implementar no Projeto Integrador, o desenvolvimento de um Software baseado no gerenciamento de ordens de manutenção.

% CAP 1.5
% ---


% CAP 1.6
% ---
\section{Método de Trabalho}
% ---
Para o desenvolvimento do projeto será utilizado a metodologia SCRUM no formato MVP (Produto Mínimo Viável).

O Scrum, criado em 1993 por Ken Schwaber e Jeff Sutherland, tem a origem de seu nome no “jogo de rúgbi e se refere à maneira como um time trabalha junto para avançar com a bola no campo. Alinhamento cuidado, unidade de propósito, clareza de objetivo, tudo se unindo \cite{rocha2015metodologia}.
A metodologia SCRUM consiste em quebrar o sistema em várias partes pequenas e fazer entregas a cada ciclo, que normalmente possuem de 1 a 2 semanas.
Enquanto o formato MVP prega o desenvolvimento de algo com o menor investimento possível, a fim da validação da ideia ou conceito utilizado.

% CAP 1.7
% ---
\section{Organização do Trabalho}
% ---
Este documento se dará da seguinte maneira: será feito uma descrição geral do sistema, os requisitos do sistema, a análise e design e a implementação do sistema.

% CAP 1.8
% ---
\section{Glossário}
SCRUM:		Metodologia ágil de desenvolvimento de projetos.

MVP:		Produto com o mínimo valor possível, visado para validação da ideia do projeto.

API:		Interface para comunicação entre diferentes aplicações.

ORM:		Tecnologia que auxilia o gerenciamento do banco de dados através da modelagens de classes.

Express:	Tecnologia que abstrai requisições web.

Sequelize:	Biblioteca de ORM para bancos relacionais, incluindo SQL Server.

Feedback:	Retorno a um acontecimento.

Software:	Programa de computador.

UC:			Unidade Curricular.

EAP:		Estrutura Analítica do Projeto.

PMBOK:      Conhecimento em Gerenciamento de Projetos.

% ---