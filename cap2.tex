\chapter{Fundamentação Teórica}

\section{Sistemas Computacionais}

Os sistemas computacionais tomaram conta da sociedade atual, o uso de tecnologias computacionais vem tendo muito espaço na comunidade, seja fazendo uso de um computador pessoal, smartphone ou um tablet. Sendo assim, tais dispositivos foram criado essencialmente para satisfazer as necessidades das empresas de forma a garantir e resolver seus problemas organizacionais, ou seja, a criação de uma ferramenta que possa armazenar e cruzar dados e informações sem que sejam necessárias pilhas de papéis. Estes sistemas garantem às empresas a possibilidade de atingir mercados em locais mais distantes. \cite{MATTIOLI2020}.

Destaca-se, que o sistema da informação é um tipo especializado de sistema, que pode ser definido de várias formas. Com isso, pode-se entender que sistemas são séries de elementos ou componentes inter-relacionados que coletam entrada de dados, manipulando-os, processando-os, disseminando a saída dos dados e informações, fornecendo assim um mecanismo de \textit{feedback}. \cite{STAIR2008}.

Atualmente a palavra \textbf{sistema} é mal utilizada, usa-se de forma indiscriminada e sem qualquer tipo de fundamento, ou ainda, é usada para expressar determinadas situações dentro de um software, principalmente nos meios empresariais conforme explica \cite{ROSSINI2006}.

Diante disso, é importante conceituar dados, sendo estes como fatos básicos, por exemplo, o nome e a quantidade de horas trabalhadas em uma semana de um funcionário, quantidade de peça em estoque ou pedidos. Importante mencionar que as informações são compostas por um conjunto de fatos organizados de modo a terem valor adicional, além do valor dos fatos propriamente ditos. Portanto, quando dados são organizados ou alcançados de maneira significativa, se transformam em informações. \cite{STAIR2008}.

Conforme informações acima dispostas, apesar de o termo sistema estar muito difundido e utilizado muitas vezes de forma leviana, seu significado é bastante preciso. Um sistema é um conjunto de elementos que trabalham de forma integrada a atingir uma ou mais finalidades.
Para que o sistema funcione corretamente, é necessário transformar dados em informações de forma que seus objetivos sejam alcançados, desde a finalidade única de cada elemento até a totalidade das funcionalidades integradas do mesmo.

No próximo capitulo será abordado a computação verde.

\section{Computação Verde}

Computação ou TI Verde é um conjunto de iniciativas de investimento na implantação, uso e gerenciamento que tem a finalidade de minimizar o impacto negativo ambiental na área de tecnologia da informação. \cite{CQP_MMD_2020}.

O investimento de capital em computação verde é composto por três dimensões: estrutural, humano e relacional.
O capital estrutural refere-se a infraestrutura da TI que engloba \textit{hardware}, \textit{software}, redes e tecnologia da informação. O capital humano é a capacidade e experiência dos profissionais de TI a respeito de conservação de energia em tecnologia e desenvolvimento pessoal com capacidades em TI verde, obtida através de treinamento e estudos. Por último, o capital relacional envolve o gerenciamento da TI verde e a relação das organizações com seus parceiros e usuários, implementando conceitos de proteção ambiental em produtos e serviços. \cite{CQP_MMD_2020}.

Com isso, a adoção das práticas de TI verde propicia uma operação mais sustentável às organizações, gerando economia com energia, papel, água, transporte, espaço físico, manutenção e descarte, proporcionando assim valor tanto para a organização quanto para a sociedade \cite{TallesMoura2017}.

No próximo capitulo será abordado o planejamento e gerenciamento de projetos de software.

\section{Planejamento e Gerenciamento de Projetos de Software}

Um projeto é um empreendimento temporário que objetiva criar um produto, resultado ou serviço único, que no caso de projeto de software é o sistema em funcionamento \cite{Julia_Mara_2018}.

Os projetos de software apresentam particularidades principalmente por desenvolver produtos incompreensíveis e pela dificuldade do seu gerenciamento, além da falta de comunicação entre os gerentes/desenvolvedores e os clientes/usuários \cite{Prado1999}. As etapas de desenvolvimento seguem um ciclo de vida com fases próprias, tais como a especificação de requisitos, análise, projeto, implementação, testes e implantação \cite{Ralf_Teresa_Erica2018}.

A competitividade e os avanços tecnológicos provocaram um aumento na exigência de qualidade e na complexidade de decisões administrativas. Para obter êxito em um projeto é necessário planejar e gerenciar com eficiência a execução de diversas atividades independentes, pois o grau de risco e incerteza quanto ao sucesso do projeto são elevados \cite{Maria_Isabel_2001}.

Para gerir um projeto, é possível utilizar modelos mais prescritivos como o PMBoK, mais adaptativos como o Scrum ou híbridos.
Os modelos prescritivos possuem uma estrutura formal de elementos do processo como atividades e tarefas, um fluxo de trabalho que descreve como cada um destes elementos deve ocorrer e como eles são relacionados. A busca pela estrutura e a ordem é uma característica importante deste tipo de modelo, que normalmente são compostos por \textit{frameworks} como PMBoK, PRINCE2 e IPMA\cite{Julia_Mara_2018}.
Enquanto os modelos prescritivos presam principalmente pela metodologia e pelo fluxo dos processos, modelos adaptativos como os oriundos do manifesto ágil, são focados na interação entre os usuários no processo de construção do produto através de uma abordagem iterativa e adaptativa\cite{Julia_Mara_2018}.
Os métodos ágeis possuem valores fundamentais, que foram definidos no manifesto ágil, de 2001:
\begin{itemize}
	\item Indivíduos e interação entre eles mais que processos e ferramentas;
	\item Software em funcionamento mais que documentação abrangente;
	\item Software em funcionamento mais que documentação abrangente;
	\item Responder a mudanças mais que seguir um plano.
\end{itemize}

\begin{figure}[htb]
	\caption{\label{planejamento_21}Planejamento de Projeto}
	\begin{center}
		\includegraphics[scale=0.45]{./Figuras/planejamento_projeto.png}
	\end{center}
	\legend{Fonte: \cite{Maria_Isabel_2001}}
\end{figure}

A figura \ref{planejamento_21} mostra o ciclo de gerenciamento de projetos de software. O controle atua tanto no planejamento quanto no desenvolvimento, proporcionando desvios que originam ações corretivas. Entretanto, a avaliação ao final do projeto, reabastece o planejamento com novos projetos e ideias, prosseguindo assim em ciclo indeterminado. Já os padrões são definidos a partir dos controles e das avaliações para controlar e planejar o decorrer do projeto.

No próximo capitulo será abordado a UML.

\section{UML - \textit{Unified Modeling Language}}
A UML foi desenvolvida com o propósito de fornecer a arquitetos e engenheiros, ferramentas para análise, design e implementação de software, bem como oferecer suporte à modelagem de negócio \cite{gasparini2018driv}. A UML é dividido em dois grupos: estrutural que permite representar a parte estática do sistema e comportamental que permite representar objetos dinâmicos no sistema \cite{silva2018sasml}.

Os modelos estruturais utilizam classes de objetos e seus relacionamento, os relacionamentos importantes que podem ser documentados nesse estágio são os de generalização e composição. Enquanto os modelos dinâmicos mostram as interações entre os objetos do sistema, as interações que podem ser documentadas incluem a sequência de solicitação de serviço feitas pelos objetos e as mudanças de estado que são disparadas por essas interações de objetos \cite{sommerville2011software}.

A figura \ref{Exemplo_UML} representa um modelo de diagrama de classes que aborda uma experiência em um \textit{e-commerce}. Um cliente pode fazer o login na plataforma, inserir, excluir e alterar produtos e finalizar a compra. Cada item do carrinho possui o produto e a quantidade, e o produto possui preço e descrição.
Após finalizar o carrinho a plataforma gera um pedido que gera um pagamento.
\begin{figure}[htb]
	\caption{\label{Exemplo_UML} Exemplo de modelo conceitual especificado em UML}
	\begin{center}
		\includegraphics[scale=0.45]{./Figuras/Exemplo_UML.png}
	\end{center}
	\legend{Fonte: \cite{monica_diagrama}}
\end{figure}

No próximo capitulo será abordado a metodologia ágil de desenvolvimento SCRUM.
\section{SCRUM}
O Scrum, criado em 1993 por Ken Schwaber e Jeff Sutherland, tem a origem de seu nome no jogo de rúgbi e se refere à maneira como um time trabalha junto para avançar com a bola no campo. Tudo se alinha: posicionamento cuidadoso, unidade de propósito, clareza de objetivo, tudo se unindo \cite{rocha2015metodologia}.

O Scrum é um \textit{framework} utilizado em projetos ágeis para o gerenciamento e desenvolvimento de produtos, tendo como foco a entrega de valor de um negócio no menor tempo possível \cite{cruz2013scrum}. Baseia-se em aproveitar a maneira como as equipes trabalham de fato, fornecendo ferramentas para se auto organizarem e otimizarem a velocidade e qualidade do trabalho \cite{sutherland2016scrum}.

Por conta disso, o Scrum vem sendo adotado por organizações de diversos tamanhos e tipos, desde multinacionais à \textit{startups}, de famosas a desconhecidas. É utilizado em projetos de características igualmente diversificadas, projetos críticos de centenas de milhares de dólares e em projetos internos e simples \cite{sabbagh2014scrum}.

No próximo capitulo será abordado a tecnologia GIT.
\section{GIT}
O Git é um sistema de controle de versão (SCM) distribuído e um sistema de gerenciamento de código fonte. Foi projetado e desenvolvido por Linus Torvalds, o criador do Linux \cite{souzagerenciamento}.
O Git permite que uma equipe de desenvolvedores possam trabalhar de forma colaborativa em um projeto sem a preocupação com perda de informação na edição ou criação de arquivos \cite{konnorate2019importancia}.
Apesar de ser distribuído e não depender de um servidor central, normalmente existe um servidor principal chamado de \textit{origin}, a partir dele os desenvolvedores do projeto clonam e trabalham nesse repositório sem necessitar mais da comunicação com o repositório principal até ser necessário sincronizar as mudanças novamente \cite{cunha2018entendendo}.
Enquanto os desenvolvedores trabalham em suas atividades, mudanças paralelas podem ocorrer, resultando em conflitos ao integrar o código. Resolver estes conflitos não é uma tarefa trivial, podendo se tornar custosa ao desenvolvedor. Estes tipos de problemas não possuem uma solução exata, pois existe mais de uma maneira de solucioná-los \cite{cunha2018entendendo}.

No próximo capitulo será abordado a técnica ORM.
\section{ORM - \textit{Object-Relational Mapping}}

ORM é um método de programação para troca de dados entre tipos de sistemas incompatíveis de banco de dados relacionais e linguagem orientada a objetos. A tecnologia ORM constitui um ``banco de dados orientado a objetos virtual'' que consegue lidar internamente com a linguagem de programação \cite{nazario2019detecting}. O seu principal objetivo é reduzir a impedância da programação orientada a objetos na utilização de bancos de dados relacionais, nela, as tabelas do banco de dados são mapeadas em classes e os registros de cada tabela são mapeados em instâncias das classes correspondentes \cite{michalsky2012componentes}, de forma que o objeto não precisa saber nada a respeito do banco de dados e nem o banco de dados ter algum conhecimento em relação ao objeto ou a estrutura de dados da aplicação, toda a conversão é realizada pela ORM \cite{fayyaz2014performance}.
O uso de \textit{frameworks} ORM reduzem significativamente o esforço de não apenas comunicar com o banco de dados mas também de lidar com operações básicas do mesmo, como inserir, atualizar, ler e deletar registros, já que as alterações de objetos são propagas automaticamente aos registros correspondentes na base de dados \cite{nazario2019detecting}.