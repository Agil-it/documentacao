\chapter{ Testes de Usabilidade}

DEFINIR USABILIDADE

Uma das definições mais encontradas na literatura é a de que a usabilidade é parte de um projeto mais amplo, que aponta para o desenvolvimento de métodos e técnicas que podem incorporar considerações de ergonomia dentro do processo de design e avaliação da interface homem/computador \cite{bastien1993ergonomic}.

Finalizar com uma Referência essa introdução. 

Para que serve a usabilidade?  3 -  Referências

Fechar com a  importância da aplicação dela no projeto Integrador.


\subsection{Elaboração dos Testes Aplicados}

A usabilidade é uma qualidade de uso, ou seja, ela é definida ou medida para um determinado contexto no qual um sistema é operado. Assim, um sistema pode proporcionar boa usabilidade para um usuário experiente, mas péssima para um iniciante, ou vice-versa; ou ainda, pode ser fácil operar se o sistema for usado esporadicamente, mas difícil se for utilizado frequentemente \cite{cybis2003engenharia}.

Para isso, estudos como os de \cite{bastien1993ergonomic}, apresentam regras e recomendações para que os sistemas computacionais sejam elaborados de modo a facilitar sua aprendizagem e uso, proporcionando usabilidade.

Para a avaliação do projeto, foi elaborado uma metodologia de verificação de alguns requisitos visando a análise e a usabilidade padrão mínimo para a implantação e utilização do web/mobile. Para isso foi criado algumas situações de usabilidade segundo alguns trabalhos científicos como a de \cite{silva2016principios}, onde verifica-se alguns critérios importantes que foram formulados para a aplicação das avaliações do desenvolvimento dos projetos Integradores. As avaliações aconteceram de forma remota com uma amostragem de aproximadamente 19 usuários, incluindo os colaboradores da empresa Duas Rodas.

Para a elaboração dos resultados, foram realizadas as médias de todas as avaliações e unificadas graficamente, buscando descreve-las o mais transparente possível. Para isso, foi criado 3 módulos de avaliação que são eles: Critério Geral de Apresentação, Critérios Gerais e Erros comuns, onde respectivamente encontra-se como características de layout, características de desenvolvimento e erros que podem ocorrer no projeto.

\subsection{Discussão dos Resultados}

Verificou-se no Gráfico \ref{criterio-geral-apresentacao}, de Critérios Gerias de apresentação do projeto Agil-it que o percentual de avaliação, um aproveitamento de 84,5\% e média de todos os resultados avaliados de 8,5, considerando os critérios de apresentação dos Layouts, onde as notas foram de 1-10 avaliando cada quesito, como observa-se no gráfico.

\begin{figure}[htb]
	\caption{\label{criterio-geral-apresentacao}Critério Geral de Apresentação}
	\begin{center}
		\includegraphics[scale=0.60]{./Figuras/cap-8/criterio-geral-apresentacao.jpeg}
	\end{center}
\end{figure}

Para a análise dos Critérios Gerais, observa-se no gráfico \ref{criterios-gerais} que o foco principal foi no desenvolvimento e desempenho do software considerando características fundamentais para o sucesso da usabilidade e da aprovação do Usuário. Neste quesito, a avaliação ficou em um aproveitamento de 88,2\% com a média de 8,8 dos quesitos avaliados.

\begin{figure}[htb]
	\caption{\label{criterios-gerais}Critérios Gerais}
	\begin{center}
		\includegraphics[scale=0.60]{./Figuras/cap-8/criterios-gerais.jpeg}
	\end{center}
\end{figure}

Quanto a avaliação dos Erros Comuns, foi avaliado com aplica-se ou não este quesito, e observa-se na tabela XXI que, o software de gerenciamento do Agil.It, chegou a um percentual médio de aprovação de 92\%, sendo assim, considera-se ótimo o percentual obtido na avaliação, na etapa do projeto.

\begin{table}[]
	\begin{tabular}{|l|c|l|l|l|l|l|}
		\hline
		& \multicolumn{5}{c|}{\textbf{Erros   comuns de Usabilidade}}                                  &                                 \\ \hline
		1                        & \multicolumn{5}{c|}{Ícones   e Menus ambíguos}                                               & \multirow{10}{*}{\textbf{92\%}} \\ \cline{1-6}
		2                        & \multicolumn{5}{c|}{Linguagens   que permitem apenas movimentos direcionados de forma única} &                                 \\ \cline{1-6}
		3                        & \multicolumn{5}{c|}{Limite   de entrada e manipulação}                                       &                                 \\ \cline{1-6}
		4                        & \multicolumn{5}{c|}{Limite   de seleção e destaque}                                          &                                 \\ \cline{1-6}
		5                        & \multicolumn{5}{c|}{Sequência   não clara de passos}                                         &                                 \\ \cline{1-6}
		6                        & \multicolumn{5}{c|}{Mais   passos para gerenciar a interface do que para realizar tarefas}   &                                 \\ \cline{1-6}
		7                        & \multicolumn{5}{c|}{Links   complexos entre/com aplicações}                                  &                                 \\ \cline{1-6}
		8                        & \multicolumn{5}{c|}{Confirmações   e Feedbacks inadequados}                                  &                                 \\ \cline{1-6}
		9                        & \multicolumn{5}{c|}{Pouca   inteligência e antecipação por conta do sistema}                 &                                 \\ \cline{1-6}
		10                       & \multicolumn{5}{c|}{Mensagem   de erro, tutoriais ajuda e documentação inadequadas}          &                                 \\ \hline
		\multicolumn{1}{|c|}{11} & \multicolumn{5}{c|}{Poluição Visual}                                                         & \multirow{6}{*}{}               \\ \cline{1-6}
		\multicolumn{1}{|c|}{12} & \multicolumn{5}{c|}{Má Organização da Informação}                                            &                                 \\ \cline{1-6}
		\multicolumn{1}{|c|}{13} & \multicolumn{5}{c|}{Componentes Incompreensíveis}                                            &                                 \\ \cline{1-6}
		\multicolumn{1}{|c|}{14} & \multicolumn{5}{c|}{Distrações irritantes}                                                   &                                 \\ \cline{1-6}
		\multicolumn{1}{|c|}{15} & \multicolumn{5}{c|}{Navegação Ineficiente}                                                   &                                 \\ \cline{1-6}
		\multicolumn{1}{|c|}{16} & \multicolumn{5}{c|}{Sobrecarga de Informações}                                               &                                 \\ \hline
	\end{tabular}
\end{table}