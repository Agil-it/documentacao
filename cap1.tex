\chapter{Introdução}

{A faculdade SENAI (Serviço Nacional de Aprendizagem Industrial) \footnote{\url{http://sc.senai.br/pt-br/faculdade-senai-jaragua-do-sul}}  visa sistematizar os conhecimentos adquiridos pelos estudantes durante o decorrer do curso, como também, oferecer vivência  prática-profissional mediante a aplicação dos conhecimentos em situações reais. Além disso, busca  propiciar ao estudante o contato com o universo acadêmico da iniciação científica, através da implantação do \textit{Projeto Integrador}, que tem exatamente essas características em locais que está sendo implantado. Portanto, o Senai está em parceria com a empresa Duas Rodas desenvolvendo este projeto. }

A empresa Duas Rodas possui sua sede em Jaraguá do Sul, Santa Catarina e tem como principal ramo a indústria de alimentos \footnote{\url{ https://www.duasrodas.com/}}. Para a execução de suas tarefas diárias, seus funcionários contam com o auxílio direto de máquinas de pequeno, médio e grande porte, e sabe-se que esses equipamentos necessitam de manutenção preventiva, preditiva e corretiva  de tempos em tempos. Para tanto, se faz necessário ajustes adequados e rápidos para que não haja baixa produtividade e consequentemente uma redução em seus proventos.

Esses equipamentos precisam passar por uma série de cuidados e estar em bom estado para o uso dos seus colaboradores. Portanto, é de extrema importância para a empresa Duas Rodas ter um controle e um bom fluxo de manutenções desses equipamentos, de modo a não haver uma inatividade desnecessária, e quando necessário a manutenção, esta seja rápida e eficiente.

Entendendo o cenário acima, este trabalho se propõe a melhorar o processo de manutenção de equipamentos da empresa Duas Rodas, através de um sistema para gerenciamento de ordens de manutenção que irá linkar a parte administrativa com o operacional, fazendo com que o administrador possa ter uma comunicação rápida com o técnico e visualizar em tempo real o andamento da manutenção. Sendo assim, este processo ficará mais eficiente e totalmente digital, fazendo com que a empresa reduza a quantidade de papel impresso, gerando assim uma economia e tornando-a mais sustentável.

\section{Problema}

O processo de manutenção de equipamentos de um parque fabril necessita de um acompanhamento constante das tarefas que serão executadas. No cenário da empresa Duas Rodas, o acompanhamento das ordens de manutenção ocorre de forma impressa, o que torna difícil a gestão da mesma, pois necessita que as informações sejam transportadas fisicamente entre os colaboradores, gestores e técnicos. No final da manutenção é preciso que seja digitado ao ERP SAP todo o processo executado pelos técnicos nos equipamentos.

\section{Objetivos Gerais}

O presente estudo visa aperfeiçoar os processos administrativos no que diz respeito às ordens de manutenção feitas pela empresa Duas Rodas, através de um sistema digital que irá facilitar a execução da manutenção de equipamentos e a gestão do processo, desde a abertura de uma ordem até o seu encerramento.

\section{Objetivos Específicos}

\begin{itemize}
	\item Identificar referências bibliográficas voltadas a softwares e gestão de manutenção.
	\item Elaborar a análise dos requisitos e a prototipação do sistema.
	\item Desenvolver o sistema.
	\item Analisar e validar o desenvolvimento do sistema.
	\item Testes necessários.
	\item Apresentar o sistema desenvolvido conforme o problema proposto.
\end{itemize}


\section{Justificativa}

O desenvolvimento do projeto irá tornar o processo de manutenção de equipamentos da empresa Duas Rodas mais eficiente, através de um sistema de gerenciamento de manutenção que irá interligar toda a parte administrativa ao setor operacional, deixando o processo mais fluído, irá também reduzir os gastos com papeis e o tempo demandado para a gestão do processo.

O ciclo de vida da informação, é a mudança no valor da informação com o decorrer do tempo. Quando os dados são criados, muitas vezes possuem seu valor mais alto e são usados com frequência. Porém, conforme o tempo passa, os dados não digitais se perdem com mais facilidade e tem menos valor para a organização.  As empresas modernas precisam que seus dados estejam protegidos, íntegros e disponíveis em tempo integral. Com isso, os sistemas digitais podem fornecer a otimização apropriada de armazenamento, uma política eficaz de gerenciamento de dados necessária para dar suporte e potencializar os benefícios da empresa, explica \cite{somasundaram2009armazenamento}.

As empresas não devem apenas beneficiar os proprietários, mas toda a sociedade, especialmente as classes mais penalizadas. Não basta somente a responsabilidade social, pois a comunidade deve ser pensada junto a interface com a natureza, da qual é um subsistema. Com isso, se introduziu a responsabilidade socioambiental, programas que tem por objetivo diminuir a pressão que a atividade produtiva e industrialista faz sobre o meio ambiente. As inovações tecnológicas ajudam neste propósito sem mudar o rumo do crescimento e desenvolvimento que implica a dominação da natureza, de acordo com \cite{boff2017sustentabilidade}.

Exposto isso, o sistema proposto objetiva a diminuição do consumo excessivo de papel no processo de manutenção de equipamentos, deixando a empresa Duas Rodas mais sustentável e preparada para as tendências futuras envolvendo o meio ambiente.

% CAP 1.6
% ---
\section{Método de Trabalho}
% ---
Para o desenvolvimento do projeto foi utilizado a metodologia SCRUM no formato MVP (Produto Mínimo Viável).
Essa metodologia consiste em quebrar o sistema em várias partes pequenas e fazer entregas a cada ciclo, que normalmente possuem de 1 a 2 semanas.
E o formato MVP prega o desenvolvimento de algo com o menor investimento possível, a fim da validação da ideia ou conceito utilizado.

% CAP 1.7
% ---
\section{Organização do Trabalho}
% ---

O desenvolvimento do trabalho será composto pela fundamentação teórica, descrição geral do sistema, pesquisa de anterioridade, requisitos do sistema, planejamento do sistema, a implementação do sistema e por fim, testes de usabilidade do sistema.

% CAP 1.8
% ---
\section{Glossário}
SCRUM:		Metodologia ágil de desenvolvimento de projetos.

MVP:		Produto com o mínimo valor possível, visado para validação da ideia do projeto.

API:		Interface para comunicação entre diferentes aplicações.

ORM:		Tecnologia que auxilia o gerenciamento do banco de dados através da modelagens de classes.

Express:	Tecnologia que abstrai requisições web.

Sequelize:	Biblioteca de ORM para bancos relacionais, incluindo SQL Server.

Feedback:	Retorno a um acontecimento.

Software:	Programa de computador.

EAP:		Estrutura Analítica do Projeto.

PMBOK:      Conhecimento em Gerenciamento de Projetos.

ERP:        Software de gerenciamento de empresas.

SAP:        ERP alemão conhecido mundialmente.

Front-end:	Parte visual do sistema.

Back-end:	Parte lógica do sistema.

CRUD:	  	Definição de uma rotina capaz de cadastrar, ler, atualizar e deletar registros de uma tabela.

% ---





