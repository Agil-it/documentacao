\chapter{Implementação}

Nessa etapa, o sistema codificado a partir da descrição computacional das fases de projeto, como visto nos capítulos anteriores, conta com três aplicações: WEB, Mobile e Servidor. As aplicações foram desenvolvidas ao longo de três semestres no qual disponibilizou-se períodos de implementação e apresentação aos envolvidos, sendo estes a Faculdade de Tecnologia Senai e a Indústria Duas Rodas com o objetivo de alinhar o desenvolvimento e as expectativas do software em relação ao processo de manutenção executado atualmente.

\section{Tecnologias}
Para o desenvolvimento do sistema, foram divididas em três aplicações: uma visando a parte WEB, outra visando dispositivos móveis com os sistemas Android e IOS e para efetuar a comunicação entre os sistemas e o banco dados, foi elaborado uma aplicação de servidor cujo valida as requisições e efetua a comunicação entre as aplicações visíveis e o banco de dados.

\subsection{Aplicação WEB}

Para o desenvolvimento foi utilizada as tecnologias pré requisitadas pela empresa Duas Rodas, sendo elas HTML5\footnote{\url{https://html.spec.whatwg.org/}}, SASS\footnote{\url{https://sass-lang.com/}} e como complemento, utilizamos a biblioteca ReactJS\footnote{\url{https://reactjs.org/}}.

{\textbf{HTML5} - É uma linguagem de marcação utilizada para desenvolvimento Web, esta nova versão traz consigo importantes mudanças quanto ao papel do HTML no mundo da Web, através de novas funcionalidades como semântica e acessibilidade.}

{\textbf{SASS} - É uma folha de estilo interpretada e compilada em Cascading style sheets, ao ser interpretado é criado blocos de códigos de regras CSS. Resumidamente é utilizada para fazer o Design do sistema, com cores personalizadas, etc.}

{\textbf{ReactJs} - O React é uma biblioteca JavaScript de código aberto com foco em criar interfaces de usuário em páginas web.}

\subsection{Aplicação Mobile}
Utilizando as tecnologias Ionic 4\footnote{\url{https://ionicframework.com/}} e TypeScript\footnote{\url{https://www.typescriptlang.org/}} esta aplicação é desenvolvida paralelamente com as demais aplicações, tem como objetivo a mobilidade, facilidade de acesso e interação podendo ser utilizado em qualquer lugar dentro da empresa.

{\textbf{TypeScript} - TypeScript é um superconjunto de JavaScript desenvolvido pela Microsoft que adiciona tipagem e alguns outros recursos a linguagem. É utilizada somente pelos desenvolvedores, pois auxilia na construção do código fonte. Seu código é transpilado para JavaScript, no final das contas tudo escrito em TypeScript vira JavaScript.}

{\textbf{Ionic 4} - O Ionic é um \textit{framework} de código aberto para desenvolvimento de aplicativos móveis multiplataforma.}

\subsection{Aplicação do Servidor}
A aplicação do servidor é o canal centralizador das demais plataformas. Ela tem acesso único e exclusivo ao banco de dados. É responsável por todas as validações e executa regras de negócio.
O servidor foi desenvolvido em Typescript e utiliza a TypeOrm\footnote{\url{https://typeorm.io/}} para integração com o banco de dados, para a API utiliza o Express\footnote{\url{https://expressjs.com/}}.

{\textbf{TypeOrm} - TypeOrm é uma ORM para Typescript, utiliza injeção de dependências para modelar as classes e transformar em tabelas.}

{\textbf{Express} - Express é uma biblioteca que abstrai a camada HTTP e gerencia requisições recebidas pelo protocolo.}


\section{Códigos Desenvolvidos} 

Nesta etapa será apresentado alguns trechos dos códigos implementados para ilustrar o funcionamento e o entendimento de partes específicas do código fonte, onde os mesmos contém regras complexas e de suma importância ao leitor.

\subsection{WEB}
\subsubsection{Componentes}
A plataforma web da Agil.It tem como padrão componentizar os elementos exibidos no \textit{front-end}, para isso, utiliza-se a biblioteca \textit{React-md}, com componentes acessíveis por propriedades (parâmetros) e totalmente personalizáveis. Os estilos podem ser configurados tanto em tempo de compilação quanto em tempo de execução pelas variáveis SASS configuráveis e pelo uso de CSS.
A biblioteca deve ser instalada no projeto e importada no código fonte conforme é mostrado no código \ref{react-md-components}.

\begin{lstlisting}[language=JavaScript, caption={Importando o componente de Ícone do react-md}, label={react-md-components}]
import React, { PureComponent } from 'react';
import { FontIcon } from 'react-md';

export class C_Icon extends React.Component {
	constructor(props) {
		super(props);
	}
	
	render() {
		return (
			<FontIcon 
				className={this.props.className}
				primary={this.props.primary}
				forceFontSize={!!this.props.iconSize}
				secondary={this.props.secondary}
				forceSize={this.props.iconSize}
				label={this.props.label}
				style={this.props.style}
				onClick={this.props.action}
				disabled={this.props.disabled}
				name={this.props.name}
			>
				{this.props.icon}
			</FontIcon>
		);
	}
}
\end{lstlisting}

\subsubsection{Cookies}
Ao efetuar o login na plataforma, as informações do usuário, exceto sua senha, são armazenadas nos \textit{cookies} através da biblioteca \textit{universal-cookie} e são manipuladas para ocultar telas de determinados usuários de acordo com o seu perfil ou exibir alguma tela adicional de controle e análise, caso o perfil logado seja de um administrador. Outro exemplo seria impedir/permitir que tal usuário possa executar determinada ação ocultando elementos na tela de acordo com seu perfil.
Com isso, o controle de informações e ações que o sistema oferece se torna muito mais fácil.

Após cada requisição feita ao servidor, o mesmo retorna um novo \textit{token} renovado e então atualiza-se o \textit{token} e usuário do \textit{cookie}.
Conforme demonstrado no código \ref{storage-cookie-data}

\begin{lstlisting}[language=JavaScript, caption={Gravar dados no cookie}, label={storage-cookie-data}]
setToken(token) {
	const user = JWTHelper.decomposeJwt(token)

	this.cookies.set('token', token, { path: '/' })
	this.cookies.set('user', user, { path: '/' })
}
\end{lstlisting}

Para recuperar os valores dos \textit{cookies} basta instanciar o \textit{universal-cookies} e utilizar o método \textit{get}. Para remover o dado basta utilizar o método \textit{remove}, de acordo com o código \ref{universal-cookie-methods}.

\begin{lstlisting}[language=JavaScript, caption={Recuperar dados do cookie}, label={universal-cookie-methods}]
class App extends Component {
	constructor() {
		super()

		this.cookies = new Cookies();

		this.state = { 
			token: this.cookies.get('token'),
			user: this.cookies.get('user'),
		};
	};
	
	onLogout() {
		this.cookies.remove('token');
		this.setState({ token: undefined })
	}
}
\end{lstlisting}

\subsubsection{Helpers}
Com o objetivo de agilizar o desenvolvimento do projeto, foram implementadas diversas classes de ajuda como \textit{DateHelper}, \textit{MaintenanceOrderHelpers} e \textit{SearchModel}.

\textit{DateHelper} é responsável por manipular e retornar as datas no formato desejado, possui funções que podem ser chamadas em qualquer classe para auxiliar o time de desenvolvimento no ganho de tempo. Como manipular datas é trabalhoso, segue um exemplo de como deixar um pouco mais simples, como pode-se ver no código \ref{date-helper}.

\begin{lstlisting}[language=JavaScript, caption={Função que formata a data em dia/mês/ano - horas/minutos}, label={date-helper}]
static formatDateTime(inputDate) {
	var date = this.getDate(inputDate);
	
	var day = StringHelper.JustifyLeft(date.getDate(), 2, 0);
	var month = StringHelper.JustifyLeft(date.getMonth() + 1, 2, 0);
	var year = date.getFullYear();
	var hour = date.getHours();
	var minutes = date.getMinutes();
	
	return `${day}/${month}/${year} às ${hour}h${minutes}`;
}
\end{lstlisting}

A \textit{MaintenanceOrderHelper} é responsável por auxiliar na manipulação dos dados referentes as ordens de manutenção, possui funções que fazem a tradução de algumas propriedades dos objetos que são salvos em inglês no servidor.

Código \ref{maintenance-order-helper-translate-props} exemplifica a utilização da função de tradução de propriedades da ordem na classe de Ordem de Manutenção.

\begin{lstlisting}[language=JavaScript, caption={Utilizando a tradução de propriedades da ordem de manutenção}, label={maintenance-order-helper-translate-props}]
<span>{HelperOM.translate("status", order.orderStatus)}</span>
<span>{HelperOM.translate("color", order.orderStatus)}</span>
<span>{HelperOM.translate("layout", order.orderLayout)}</span>
<span>{HelperOM.translate("priority", order.priority)}</span>
\end{lstlisting}

Código \ref{maintenance-order-helper-translate-method} demonstra a função que traduz as propriedades da Ordem.

\begin{lstlisting}[language=JavaScript, caption={Função de tradução}, label={maintenance-order-helper-translate-method}]
static translate(prop, value) {
	let props;
	
	if (prop == "priority") props = this.getPriority();
	else if (prop == "status") props = this.getStatus();
	else if (prop == "color") props = this.getColorPriority();
	else if (prop == "layout") props = this.getLayoutType();
	else return value;
	
	return props[value];
}

static getLayoutType() {
	return {
		default: 'Corretiva | Preventiva',
		route: 'ROTA',
		list: 'LISTA',
	}
}

static getColorPriority() {
	return {
		low: "#03a140",
		medium: "#3177e8",
		high: "#ffd300",
		urgent: "red",
	}
}

static getPriority() {
	return {
		low: "Baixa",
		medium: "Média",
		high: "Alta",
		urgent: "Urgente",
	}
}

static getStatus() {
	return {
		created: "Aberta",
		assumed: "Assumida",
		started: "Iniciada",
		paused: "Pausada",
		stopped: "Parada",
		canceled: "Cancelada",
		"signature-pending": "Assinatura Pendente",
		signatured: "Assinada",
		finished: "Finalizada",
		"no_status": "Sem Status",
	};
}
\end{lstlisting}

Ainda no \textit{MaintenanceOrderHelper} existem funções que ordenam as OM por prioridade e data de abertura na tela de monitoramento.
As ordens serão ordenadas das urgentes às com baixa prioridade e como um segundo fator será a data de abertura, sendo as mais antigas mostradas primeiro.

Código \ref{maintenance-order-helper-sort} mostra a lógica utilizada para a ordenação.

\begin{lstlisting}[language=JavaScript, caption={Funções responsáveis pela ordenação}, label={maintenance-order-helper-sort}]
static sortOrders(list) {
	var ordenatedPriority = this.ordenatedPriority();
	
	return list.sort((a, b) => {
		if (ordenatedPriority[a.priority] > ordenatedPriority[b.priority]) return -1;
		else if (ordenatedPriority[a.priority] < ordenatedPriority[b.priority]) return 1;
		
		var dateTimeA = DateHelper.getDate(a.openedDate).getTime();
		var dateTimeB = DateHelper.getDate(b.openedDate).getTime();
		
		if (dateTimeA > dateTimeB) return 1;
		else if (dateTimeA == dateTimeB) return 0;
		else return -1;
	});
}

static ordenatedPriority() {
	return {
		urgent: 3,
		high: 2,
		medium: 1,
		low: 0,
	};
}
\end{lstlisting}

O \textit{SearchModel} contém funções pré-configuradas das colunas que serão exibidas nas telas de CRUD afim de uma melhor visualização e organização dos dados exibidos.
A configuração é composta por um \textit{array} de objetos que possui três propriedades: \textit{name}, \textit{property} e \textit{defaultValue}.
A propriedade \textit{name} será o cabeçalho da coluna da tabela, \textit{property} é a propriedade da listagem que será renderizado na coluna e \textit{defaultValue} é o valor que será apresentado caso não haja dados para a \textit{property} na listagem.

Código \ref{search-model-data-config} mostra a definição da configuração do \textit{SearchModel}.

\begin{lstlisting}[language=JavaScript, caption={Definição da configuração para consulta de dados}, label={search-model-data-config}]
const machineTypeColumns = () => [
	{
		name: "ID",
		property:"id",
		defaultValue: "Sem ID",
	},
	{
		name: "Descrição",
		property:"description",
		defaultValue: "Sem Descrição",
	},
];

const equipmentColumns = () => [
	{
		name: "Código",
		property:"code",
		defaultValue: "Sem Código",
	},
	{
		name: "Descrição",
		property:"description",
		defaultValue: "Sem Descrição",
	},
	{
		name: "Tipo Máquina",
		property:"machineType.description",
		defaultValue: "Sem Tipo de Máquina",
	},
];
\end{lstlisting}

O código \ref{search-model-equipment} mostra a utilização das funções de busca na classe de cadastro de equipamentos.

\begin{lstlisting}[language=JavaScript, caption={Tela para cadastro de equipamento}, label={search-model-equipment}]
class CreateMachine extends Component {
	
	constructor(props) {
		super(props);
		
		this.state = {
			visible: true,
			equipmentId: '',
			machineType: '',
			fields: {},
			equipmentList: [],
			machineTypeList: [],
			equipmentColumns: equipmentColumns(),
			machineTypeColumns: machineTypeColumns(),
		};
	};
}
\end{lstlisting}

Ao obter as colunas em um \textit{state}, os dados são enviados ao componente \textit{C{\_}AutoComplete} que possui um ícone de lupa no qual o usuário pode clicar para visualizar todos os equipamentos cadastrados, conforme visto no código \ref{search-model-auto-complete}.

\begin{lstlisting}[language=JavaScript, caption={Utilizando o componente C\_AutoComplete}, label={search-model-auto-complete}]
	<C_AutoComplete
		id="id"
		name="id"
		value={this.state.equipmentId}
		label={"Equipamento"}
		rightIcon={"search"}
		list={this.state.equipmentList}
		searchColumns={this.state.equipmentColumns}
	/>
\end{lstlisting}

Código \ref{auto-complete-on-click-icon} mostra a função dentro do componente \textit{C{\_}AutoComplete} que exibe a tabela de consulta de acordo com as colunas e lista de dados enviados por parâmetro.

\begin{lstlisting}[language=JavaScript, caption={Componente C\_AutoComplete: Função executada ao clicar na lupa}, label={auto-complete-on-click-icon}]
	onClickIcon() {
		if (!Array.isArray(this.props.searchColumns)) return;
		
		confirmAlert({
			customUI: ({ onClose }) => (
				<C_SearchTable
					columns={this.props.searchColumns}
					content={this.props.list}
					onClick={this.tableSelected}
					extraFunction={onClose}
				/>
			)
		});
	}
\end{lstlisting}

\subsection{Mobile}
\subsubsection{Ações da Ordem de Manutenção}
A estrutura da classe \textit{AgilitActionUtils} é feita com o intuito de reutilizar códigos, sendo que cada tipo de OM deve ser assinada da mesma forma que qualquer outra, mudando apenas seu atributo identificador, a seguir seguem algumas características importantes desta classe.

A primeira característica desta classe é realizar as ações de uma determinada OM de forma centralizada, recebendo como parâmetro para qual OM deverá ser feita a ação e a ação propriamente dita, conforme mostra código \ref{order-actions}.

\begin{lstlisting}[language=JavaScript, caption={Alterar situação da ordem de manutenção}, label={order-actions}]
public async changeStatus(orderID, orderStatus : AgilitOrderStatus){    
	return this.restOrder.orderActions(orderID, orderStatus);
}

public orderActions(orderID : number, agilitOrderStatus : AgilitOrderStatus){
	const orderStatus : any = {
		orderStatus: agilitOrderStatus        
	}
	
	if (agilitOrderStatus == AgilitOrderStatus.ASSUMED){
		let userInfo : any = AgilitStorageUtils.getDataJSON(AgilitStorageTypes.USERDATA);
		
		orderStatus.userId = userInfo.id;
	}    
	
	this.http.url = this.http.getBaseUrl() + this.restAction + '/' + orderID + '/status';
	return ProviderHelper.put(this.http, orderStatus);
}

\end{lstlisting}

Outra característica muito importante desta classe é realizar as assinaturas de uma determinada OM, sendo passada como parâmetro a OM respectivamente e a senha de assinatura. Ao realizar uma assinatura a classe irá executar algumas validações da OM a fim de verificar se está de acordo para enviar ao servidor, com isso a promessa deve ser resolvida ou rejeitada.
O código \ref{sign-order-mobile} demonstra a implementação.

\begin{lstlisting}[language=JavaScript, caption={Assinar a ordem de manutenção}, label={sign-order-mobile}]
public async signOrder(order, assignaturePassword): Promise<any>{
	return new Promise(async (resolve, reject) => {
		this.resolvePromise = resolve;
		this.rejectPromise  = reject;
		
		try {
			if (!this.validateOrder(order, assignaturePassword)){
				return;
			}        
			
			const user = AgilitStorageUtils.getDataJSON(AgilitStorageTypes.USERDATA);
			
			if (AgilitUtils.isNullOrUndefined(user) || AgilitUtils.isNullOrUndefined(user.id) || user.id == ''){
				this.rejectPromise('Dados de usuário inválido!');
				return;
			}
			
			await this.restOrder.orderAssignature(order.id, user.id).then(
			(response: any) => {
				if (AgilitUtils.isNullOrUndefined(response)){
					return;
				}
				
				this.resolvePromise(response);
			}
			).catch(
			error => {
				this.rejectPromise(error);
			}
			);
		} catch (error) {
			this.rejectPromise(error);
		}
	});
}
\end{lstlisting}

Para que uma OM seja assinada com sucesso a classe deve conter as validações dos estados atuais da OM a ser assinada, por exemplo, se o usuário realizar a assinatura de uma OM com status cancelada esta implementação tem de validar a mesma.
Código \ref{order-validation} mostra a implementação de validação da ordem antes da assinatura.

\begin{lstlisting}[language=JavaScript, caption={Validar estados da OM}, label={order-validation}]
private validateOrder(order, password) : boolean{
	if (password != AgilitStorageUtils.getData(AgilitStorageTypes.PASSWORD)){
		this.rejectPromise('Senha incorreta!');
		return;
	}
	
	if (order.orderStatus == AgilitOrderStatus.SIGNATURED){
		this.rejectPromise('OM já está assinada!');
		return;
	}
	
	if (order.orderStatus == AgilitOrderStatus.FINISHED){
		this.rejectPromise('OM já está assinada!');
		return;
	}
	
	if (order.orderStatus == AgilitOrderStatus.CANCELED){      
		this.rejectPromise('OM está cancelada!');
		return;
	}
	
	return true;
}
\end{lstlisting}

\subsubsection{Dados Armazenados no Local Storage}
Algumas informações, por exemplo, usuário e token de acesso devem ficar salvas em memória para posteriormente serem utilizadas novamente, com isso desenvolvemos a classe chamada \textit{AgilitStorageUtils} contendo regras, limpezas específicas dos dados e separação dos tipos de dados que estão sendo armazenados e buscados.

O código \ref{AgilitStorageTypes} mostra os tipos de dados armazenados.
\begin{lstlisting}[language=JavaScript, caption={Tipos de dados armazenados}, label={AgilitStorageTypes}]
export enum AgilitStorageTypes{
	USERDATA = 'user',
	USERNAME = 'username',
	PASSWORD = 'password',
	TOKEN    = 'token'
}
\end{lstlisting}

O código \ref{AgilitStorageTypes-actions} mostra as ações de limpeza, busca e gravação de dados.

\begin{lstlisting}[language=JavaScript, caption={Tipos de dados armazenados}, label={AgilitStorageTypes-actions}]

// Remove de dados conforme o tipo especificado.
public static clearSpecificData(agilitStorageType : AgilitStorageTypes) {
	window.localStorage.removeItem(agilitStorageType);
}

// Buscando dados de um tipo específico armazenado na memória do navegador.
public static getData(agilitStorageType : AgilitStorageTypes) {
	return window.localStorage.getItem(agilitStorageType);
}

// Buscando dados de um tipo específico armazenado e convertendo para objeto.
public static getDataJSON(agilitStorageType : AgilitStorageTypes) {
	return JSON.parse(window.localStorage.getItem(agilitStorageType));
}

// Este método faz uma verificação do tipo do dado que está sendo inserido na memória do navegador.
public static setData(agilitStorageType : AgilitStorageTypes, data : Object|Array<any>|string) {
	if (AgilitUtils.isNullOrUndefined(data)){
		return;
	}
	
	if (typeof data === 'object'){
		window.localStorage.setItem(agilitStorageType, JSON.stringify(data));
		return;
	}
	
	if (typeof data === 'string'){
		window.localStorage.setItem(agilitStorageType, data);
	}    
}
\end{lstlisting}

\subsection{Servidor}
\subsubsection{Mapeamento do Banco de Dados com \textit{TypeOrm}}

A \textit{TypeOrm} trabalha com injeção de dependências, então em cada propriedade tem um \textit{decorator} que vai definir o mapeamento dela. Para gerar uma \textit{primary key} com auto increment, basta usar a injeção \textit{PrimaryGeneratedColumn}, para uma coluna simples, apenas \textit{column}, para data de criação do registro, \textit{CreateDateColumn} e para data de atualização do registro \textit{UpdateDateColumn}.
Com a \textit{TypeOrm} é possível ter classes abstratas para abranger propriedades em comum no banco de dados, para tanto foi criado a classe \textit{baseClass} que irá conter todos os campos em comum de todas as tabelas do banco de dados, conforme mostrado no código \ref{typeorm-base-class}.

\begin{lstlisting}[language=JavaScript, caption={BaseClass: Mapeamento de propriedades de tabelas do banco de dados}, label={typeorm-base-class}]
export abstract class BaseClass {
	
	@PrimaryGeneratedColumn()
	public id: number | undefined = undefined;
	
	@Column({ default: '' })
	public integrationID: string = '';
	
	@CreateDateColumn()
	public createdAt: Date | undefined = undefined;
	
	@Column()
	public createdBy: number | undefined = undefined;
	
	@UpdateDateColumn()
	public updatedAt: Date | undefined = undefined;
	
	@Column()
	public updatedBy: number | undefined = undefined;
	
	@Column({
		type: Boolean,
		default: false
	})
	public deleted: boolean = false;
	
	constructor() {
	}
	
}
\end{lstlisting}

Para facilitar as tabelas genéricas de CRUD, onde têm apenas os dados base e o campo de descrição, foi criado a classe abstrata \textit{crudClass}, conforme mostrado no código \ref{typeorm-crud-class}.

\begin{lstlisting}[language=JavaScript, caption={CrudClass: Mapeamento de propriedades para CRUDs genéricos}, label={typeorm-crud-class}]
export abstract class CrudClass extends BaseClass {

	@Column({nullable: false})
	@IsNotEmpty({
		message:'Descrição: Campo obrigatório.'
	})
	public description: string = '';

	constructor() {
		super();
	}
}
\end{lstlisting}

E por fim, o código \ref{model-sector} mostra a aplicação de uma classe que irá gerar de fato uma tabela no banco de dados.

\begin{lstlisting}[language=JavaScript, caption={Mapeamento de propriedades da tabela sector}, label={model-sector}]
@Entity('sector')
export class Sector extends CrudClass {

	constructor(){
		super();
	}

}
\end{lstlisting}

No primeiro parâmetro da injeção \textit{Entity} define-se o nome da tabela, e como a classe \textit{Sector} estende a classe \textit{CrudClass} que estende, subsequentemente a classe \textit{BaseClass} a tabela terá todas as colunas definidas nas injeções de dependências mostradas anteriormente.

Para fazer relações entre as tabelas existem alguns \textit{decorators} específicos: \textit{ManyToOne} para relações muitos para um, \textit{ManyToMany} para relações muitos para muitos, \textit{OneToOne} para relações um para um e \textit{OneToMany} para relações um para muitos.

O código \ref{model-installation-area} mostra a definição da relação \textit{ManyToOne}.

\begin{lstlisting}[language=JavaScript, caption={Mapeamento de propriedades da tabela área de installation\_area}, label={model-installation-area}]
@Entity("installation_area")
export class InstallationArea extends CrudClass {
	
	@ManyToOne(type => Sector, sector => sector.id, { nullable: false })
	@JoinColumn()
	public sector: Sector = undefined;
	
	constructor() {
		super();
	}
	
}
\end{lstlisting}

A \textit{TypeOrm} tenta fazer a relação automaticamente pelo nome dos campos, porém, se for uma relação mais complexa é possível usar o \textit{decorator JoinColumn} para definir qual coluna da tabela A se relaciona com qual coluna da tabela B e explicitar em qual das tabelas deve ficar a \textit{foreign key}, conforme mostrado no código \ref{model-maintenance-worker}.

\begin{lstlisting}[language=JavaScript, caption={Mapeamento de propriedades da tabela maintenance\_worker}, label={model-maintenance-worker}]
@Entity('maintenance_worker')
export class MaintenanceWorker extends BaseClass {

	@ManyToOne(type => User, user => user.id)
	@JoinColumn()
	public user : User;

	@ManyToOne(type => MaintenanceOrder, maintenanceOrder => maintenanceOrder.id, { cascade: false })
	@JoinColumn()
	public maintenanceOrder : MaintenanceOrder | undefined = undefined;

	@Column()
	public isMain: boolean = false;

	@Column()
	public isActive: boolean = false;

	@OneToMany(type => WorkerRequest, workerRequest => workerRequest.maintenanceWorker, { cascade: false })
	public workerRequest: Array<WorkerRequest>;

	@OneToMany(type => WorkedTime, workedTime => workedTime.maintenanceWorker, { cascade: false })
	public workedTime: Array<WorkedTime>;
}
\end{lstlisting}

\subsubsection{Validação de Objetos com o Class Validator}

Para validações de campos da tabela utiliza-se outra biblioteca com injeção de dependências que funciona muito bem quando mesclada com a \textit{TypeOrm}, o \textit{Class Validator}.
Com ele pode-se utilizar, por exemplo o \textit{decorator IsNotEmpty} na propriedade \textit{description}. Esse \textit{decorator} irá validar se a propriedade está vazia ou não, caso esteja, irá retornar o erro \textit{Descrição: Campo obrigatório.} conforme definição no próprio \textit{decorator} visto no código \ref{class-validator-model}.

\begin{lstlisting}[language=JavaScript, caption={Definição de validação de propriedades com o Class Validator}, label={class-validator-model}]
@IsNotEmpty({
message:'Descrição: Campo Obrigatório.',
})
public description: string = '';
\end{lstlisting}

E para executar a validação conforme as injeções feitas, é preciso importar a função \textit{validate} do \textit{class validator} e executar passando uma instancia do objeto.

\begin{lstlisting}[language=JavaScript, caption={Validação de Objetos com o Class Validator}]
async validate(entity: Entity) : Promise<any> {
	const errors = await validate(entity);
	
	if (errors.length === 0) {
		return undefined;
	}
	
	let errorList = [];
	
	errors.forEach(error => {
		let constraints = error.constraints;
		
		for (const key in constraints) {
			if (constraints.hasOwnProperty(key)) {
				errorList.push(constraints[key]);
			}
		}
	});
	
	return errorList;
}
\end{lstlisting}

O retorno vem em forma de \textit{array} e traz diversas informações a respeito dos erros ocorridos, no caso da API, é mostrado apenas a mensagem de erro em si, então a listagem é filtrada para mostrar apenas os erros.

\subsubsection{API}
Na API é utilizado o \textit{Express} para gerenciar as requisições feitas pela aplicação web, mobile e integração de sistemas externos.
Foi adicionado os \textit{plugins Cors} para lidar com o CORS, \textit{BodyParser} para converter JSON e o \textit{Helmet} para melhorar a segurança da aplicação.

No script de inicialização do servidor é estabelecido conexão com o banco de dados, instanciado a aplicação \textit{Express} e instalado seus \textit{plugins}, conforme visto no código \ref{script-inicial-servidor}

\begin{lstlisting}[language=JavaScript, caption={Script de inicialização do servidor}, label={script-inicial-servidor}]
createConnection().then(async connection => {

	// create express app
	const app = express();
	app.use(bodyParser.json());
	app.use(cors({exposedHeaders: 'token'}));
	app.use(helmet());

	// Get all system routes
	let routes = new Routes();

	routes.getCollections().forEach((collection: Collection) => {
		collection.getRoutes().forEach((route: Route) => {
			// Register each route
			route.registerRoute(app)
		});
	})

	
	app.listen(4000);

}).catch(error => console.log(error));
\end{lstlisting}

O roteamento é composto por várias coleções de rotas, no qual cada coleção tem um conjunto de rotas que é composto pela url que a rota responderá, os métodos que aceitará e a função que será invocada.
Por fim, o método \textit{registerRoute} irá registrar a rota da coleção ao \textit{express}.
Esses métodos que as rotas respondem ficam no \textit{package controller} que fazem a ponta entre receber a requisição, processar os dados recebidos e responder o que foi solicitado, seja cadastrar, atualizar, excluir ou consultar um ou mais registros.
Da mesma forma que foi feita com os \textit{models}, foi desenvolvido \textit{CrudController}, uma classe que utiliza \textit{Generics} que para cada instancia recebe o valor da entidade do CRUD e irá se adequar conforme o tipo fornecido. Essa classe também recebe no construtor o acesso ao repositório que será gravado, no caso, a tabela do banco de dados, conforme visto no código\ref{crud-controller}.

\begin{lstlisting}[language=JavaScript, caption={CrudController}, label={crud-controller}]
export class CrudController<Entity> {

	private repositoryEntity : Repository<Entity>;

	constructor(repositoryEntity: Repository<Entity>) {
		this.repositoryEntity = repositoryEntity;
	}
\end{lstlisting}

Todas as rotas são protegidas, com exceção da rota de login. Essa proteção é feita através de um \textit{middleware} colocado ao registrar a rota no \textit{express}, esse \textit{middleware} vai capturar os cabeçalhos \textit{token} para autenticação de sessão e \textit{authorization} para usuários de integração, caso nenhum dos cabeçalhos seja fornecido a API irá retornar o status ``401 - Unauthorized''. Caso seja passado o cabeçalho \textit{authorization} o \textit{middleware} irá verificar se chave informada é válida e se está na base de dados de usuários de integração. Se passado \textit{token}, o sistema irá decompor e validar o \textit{token} utilizando a biblioteca \textit{jsonwebtoken}, caso a validação do \textit{JWT} informado falhe, seja por não estar com a assinatura válida da aplicação ou por ter expirado o tempo da utilização, retorna um erro para a API informando que o \textit{token} não é válido. Caso a validação passe, o \textit{token} é regerado com expiração renovada para 5 horas e é adicionado ao cabeçalho da resposta. Lembrando que, o \textit{token} pode ser obtido pela primeira vez através da rota de Login.

O código \ref{middleware-check-jwt} demonstra a validação de usuário e \textit{token} feita pelo \textit{middleware} da API.
\begin{lstlisting}[language=JavaScript, caption={Função para validação de autenticação da API}, label={middleware-check-jwt}]
export const checkJwt = (request: Request, response: Response, next: NextFunction) => {
	
	const authorization = <string>request.headers["authorization"];
	
	if (authorization) {    // The Authorization was passed in so now we validate it
		this.checkIntegration(authorization)
		.then(valid => {
			if (valid === true) {
				return next();
			} else {
				return response.status(401).send();  
			}
		}).catch(err => {
			return response.status(500).send({
				"success": false,
				"error": err.message
			});
		})
	} else {
		//Get the jwt token from the request's header
		const token = <string>request.headers["token"];
		let jwtPayload;
		
		//Try to validate the token and get data
		try {
			jwtPayload = <any>jwt.verify(token, JWT.jwtSecret);
			response.locals.jwtPayload = jwtPayload;
		} catch (error) {
			//If token is not valid, respond with unauthorized
			return response.status(401).send();
		}
		
		//set to response's header the new token
		response.append('token', new UserController().generateJwtToken(<User> jwtPayload));
		
		//Call the next middleware or controller
		next();
	}
};

\end{lstlisting}


\subsubsection{Estratégia de Exclusão de Dados}

A estratégia para exclusão de dados adotado no projeto é não deletar os registros, mas alterar a coluna \textit{deleted} para \textit{true}. Desta maneira é possível rapidamente desfazer uma exclusão errada e também não ocorrerá problemas com histórico caso as entidades sejam deletadas.

Código \ref{remove-entity} mostra lógica executada durante a ``exclusão'' de um registro.
\begin{lstlisting}[language=JavaScript, caption={Método para exclusão de registro}, label={remove-entity}]
async removeEntity(entity: Entity) {
	entity["deleted"] = true;
	
	return this.getRepositoryEntity().save(entity)
}
\end{lstlisting}

