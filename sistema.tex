\chapter{O que é sistema}
Os sistemas computacionais tomaram conta da sociedade atual, o uso de tecnologias computacionais vem tendo muito espaço na comunidade, seja fazendo uso de um computador pessoal, smartphone ou um tablet. Sendo assim, tais dispositivos foram criado essencialmente para satisfazer as necessidades das empresas de forma a garantir e resolver seus problemas organizacionais, ou seja a criação de uma ferramenta que possa armazenar e cruzar dados e informações sem que sejam necessárias pilhas de papéis. Estes sistemas garantem às empresas a possibilidade de atingir mercados em locais mais distantes. \cite{MATTIOLI2020}.

Destaca-se, que o sistema da informação é um tipo especializado de sistema, que pode ser definido de várias formas. Com isso, pode-se entender que sistemas são séries de elementos ou componentes inter-relacionados que coletam entrada de dados, manipulando-os, processando-os, disseminando a saída dos dados e informações, fornecendo assim um mecanismo de \textit{feedback}. \cite{STAIR2008}.

Atualmente a palavra \textbf{sistema} é mal utilizada, usa-se de forma indiscriminada e sem qualquer tipo de fundamento, ou ainda, é usada para expressar determinadas situações dentro de um software, principalmente nos meios empresariais conforme explica \cite{ROSSINI2006}.

Diante disso, é importante conceituar dados, sendo estes como fatos básicos, por exemplo, o nome e a quantidade de horas trabalhadas em uma semana de um funcionário, quantidade de peça em estoque ou pedidos. Importante mencionar que as informações são compostas por um conjunto de fatos organizados de modo a terem valor adicional, além do valor dos fatos propriamente ditos. Portanto, quando dados são organizados ou alcançados de maneira significativa, se transformam em informações. \cite{STAIR2008}.

Conforme informações acima dispostas, apesar do termo sistema estar muito difundido e utilizado muitas vezes de forma leviana, seu significado é bastante preciso. Um sistema é um conjunto de elementos que trabalham de forma integrada a atingir uma ou mais finalidades.
Para que o sistema funcione corretamente, é necessário transformar dados em informações de forma que seus objetivos sejam alcançados, desde a finalidade única de cada elemento até a totalidade das funcionalidades integradas do mesmo.
Com base nisso, na sequência será abordado o assunto referente ao gerenciamento de ordem de manutenção.