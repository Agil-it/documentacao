% Retira espaço extra obsoleto entre as frases.
\frenchspacing 

% \pretextual

% ---
% Capa
% ---
\imprimircapa
% ---

% ---
% Folha de rosto
% (o * indica que haverá a ficha bibliográfica)
% ---
\imprimirfolhaderosto*
% ---

% ---
% Inserir a ficha bibliografica
% ---

% Isto é um exemplo de Ficha Catalográfica, ou ``Dados internacionais de
% catalogação-na-publicação''. Você pode utilizar este modelo como referência. 
% Porém, provavelmente a biblioteca da sua universidade lhe fornecerá um PDF
% com a ficha catalográfica definitiva após a defesa do trabalho. Quando estiver
% com o documento, salve-o como PDF no diretório do seu projeto e substitua todo
% o conteúdo de implementação deste arquivo pelo comando abaixo:
%
% \begin{fichacatalografica}
%     \includepdf{fig_ficha_catalografica.pdf}
% \end{fichacatalografica}
%\begin{fichacatalografica}
%	\vspace*{\fill}					% Posição vertical
%	\hrule							% Linha horizontal
%	\begin{center}					% Minipage Centralizado
%	\begin{minipage}[c]{12.5cm}		% Largura
%	
%	\imprimirautor
%	
%	\hspace{0.5cm} \imprimirtitulo  / \imprimirautor. --
%	\imprimirlocal, \imprimirdata-
%	
%	
%	\hspace{0.5cm} \imprimirorientadorRotulo~\imprimirorientador\\
%	
%	\hspace{0.5cm}
%	\parbox[t]{\textwidth}{\imprimirtipotrabalho~--~\imprimirinstituicao,
%	\imprimirdata.}\\
%	
%	
%	\end{minipage}
%	\end{center}
%	\hrule
%\end{fichacatalografica}
% ---


% ---
% Inserir errata (Se necessário)
% ---
%\begin{errata}
%Elemento opcional da \citeonline[4.2.1.2]{NBR14724:2011}. Exemplo:
%
%\vspace{\onelineskip}
%
%FERRIGNO, C. R. A. \textbf{Tratamento de neoplasias ósseas apendiculares com
%reimplantação de enxerto ósseo autólogo autoclavado associado ao plasma
%rico em plaquetas}: estudo crítico na cirurgia de preservação de membro em
%cães. 2011. 128 f. Tese (Livre-Docência) - Faculdade de Medicina Veterinária e
%Zootecnia, Universidade de São Paulo, São Paulo, 2011.
%
%\begin{table}[htb]
%\center
%\footnotesize
%\begin{tabular}{|p{1.4cm}|p{1cm}|p{3cm}|p{3cm}|}
%  \hline
%   \textbf{Folha} & \textbf{Linha}  & \textbf{Onde se lê}  & \textbf{Leia-se}  \\
%    \hline
%    1 & 10 & auto-conclavo & autoconclavo\\
%   \hline
%\end{tabular}
%\end{table}
%
%\end{errata}
% ---

% ---
% Inserir folha de aprovação
% ---

% Isto é um exemplo de Folha de aprovação, elemento obrigatório da NBR
% 14724/2011 (seção 4.2.1.3). Você pode utilizar este modelo até a aprovação
% do trabalho. Após isso, substitua todo o conteúdo deste arquivo por uma
% imagem da página assinada pela banca com o comando abaixo:
%
% \includepdf{folhadeaprovacao_final.pdf}
%
% ---

% ---
% Dedicatória
% ---
% ---

% ---
% Agradecimentos
% ---
% ---

% ---
% Epígrafe
% ---
% ---

% ---
% RESUMOS
% ---


% resumo em português
\begin{resumo}
	Atualmente, o processo de ordens de manutenção da indústria Duas Rodas é realizado manualmente, fazendo com que a empresa consuma muito papel, prejudicando o meio ambiente e como consequência, acabam se perdendo com a quantidade de informações acumuladas. Sendo assim, o projeto tem como objetivo apresentar e transformar o antigo processo em um fluxo totalmente digital com a aplicação web para gerenciar as ordens e a aplicação mobile para executá-las.
	
	\vspace{\onelineskip}
	
	\noindent
	\textbf{Palavras-chaves}: Sistema, WEB, Mobile, Ordem de Manutenção, Digital.
\end{resumo}

% resumo em inglês
\begin{resumo}[Abstract]
	\begin{otherlanguage*}{english}
		Currently, the maintenance order process for the Duas Rodas industry is carried out manually, causing the company to consume a lot of paper, damaging the environment and, as a consequence, they end up being lost with the amount of accumulated information. Therefore, the project aims to present and transform the old process into a fully digital flow with a web application to manage orders and a mobile application to execute them.
		
		\vspace{\onelineskip}
		
		\noindent 
		\textbf{Key-words}: System, WEB, Mobile, Maintenance Order, Digital.
	\end{otherlanguage*}
\end{resumo}


% ---
% inserir lista de ilustrações
% ---
\pdfbookmark[0]{\listfigurename}{lof}
\listoffigures*
\cleardoublepage
% ---

% ---
% inserir lista de tabelas
% ---
%\pdfbookmark[0]{\listtablename}{lot}
%\listoftables*
%\cleardoublepage
% ---