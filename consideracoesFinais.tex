\chapter{Considerações Finais}

%Breve resumo do tema e o que foi analisado no desenvolvimento da monografia;

%Explicar a importância do tema, qual sua relevância para o meio acadêmico, para a sociedade e até para si mesmo, como crescimento pessoal, acadêmico e profissional;

%Apresentar os resultados e a conclusão geral de sua pesquisa;

%Demonstrar se os objetivos propostos na seção de introdução da monografia foram concluídos. Se as perguntas e problemas apresentados inicialmente foram respondidas e/ou esclarecidas;

% Fazer a ligação entre as necessidas apresentadas pela duas rodas e a análise de requisitos elaborados por nós, e posteriormente aceito pela empresa.

Ao longo do desenvolvimento deste projeto, foram levantadas várias necessidades da empresa alimentícia Duas Rodas no que diz respeito ao gerenciamento de todo o seu processo administrativo e operacional das ordens de manutenção. Com base nisso, desenvolveu-se as aplicações abrangendo ao máximo os requisitos especificados pela equipe da empresa.
O projeto é de suma importância para ambas as partes envolvidas, pois a empresa se beneficiará das funcionalidades do sistema como um todo, usufruindo do seu processo automatizado e totalmente digital enquanto os desenvolvedores alcançaram um crescimento exponencial tanto na parte pessoal, acadêmica e profissional. 
Para transformar o processo de manutenção dos equipamentos mais viável para a empresa, uma das principais funcionalidades desenvolvidas no sistema foi a parte gerencial da ordem de manutenção, que permite realizar um acompanhamento detalhado e constante das tarefas a serem executadas na ordem de manutenção até que a mesma seja verificada pelo responsável da manutenção. 
O resultado de modificar o processo de execução e verificação da manutenção de forma digital gerou resultados positivos, não havendo consumo excessivo com papéis e maior agilidade na busca de manutenções específicas. Em consequência disso, impactos com meio ambiente foram reduzidos, já que não serão utilizados papéis para administrar e acompanhar as manutenções. Almejamos como desafio no final do projeto, desenvolver e integrar o sistema Agil.It com o ERP Alemão SAP.