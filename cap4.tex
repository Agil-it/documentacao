\chapter{Pesquisa de Anterioridade}
Durante a análise de requisitos, foram procurados outros sistemas que fazem esse processo de gerenciamento de ordem de manutenção para nos ajudar e entender quais são os problemas mais comuns desse ramo e quais as principais funcionalidades que o mercado oferece.

\section{Produttivo}

\textit{Produttivo} é um sistema que tem vários módulos e diferentes aplicações. Um deles é o de Ordem de Serviço de Manutenção.
Utilizando o Produttivo, um supervisor planeja e acompanha as atividades de manutenção utilizando o sistema web. Os manutentores, através de um smartphone ou tablet, alimenta dados ao programa de acordo com os defeitos identificados, bem como a solução abordada. Na finalização da ordem de serviço de manutenção, o responsável realiza a assinatura digital. \cite{produttivo}


\section{SoftByte}
\textit{Visual Machine}, da empresa SoftByte tem um fluxo bem mais simples quando comparado à Produttivo. Porém, possui algumas ferramentas interessantes, como o agendamento de manutenções preventivas e preditivas e possui um sistema de controle de investimentos em manutenção em cada um dos equipamentos. \cite{softbyte}


\section{ProdWin}
\textit{ProdWin Pw-1} em seu módulo de manutenção prove controle do tempo gasto e dos materiais utilizados nas manutenções. Também dispõe de agendamento das manutenções, podendo relacioná-las a um técnico e ser acompanhadas em tempo real. O seu sistema de manutenção preditiva adiciona automaticamente a máquina na tela de alerta de manutenções, quando aproxima-se de seu limite operacional, conforme as informações previamente cadastradas. \cite{prodwin}

\section{Diferenciais Agil.It}
Analisando os softwares encontrados, será implementado para o Agil.It o diferencial de ter integração entre sistemas, uma central de notificação e ser modelado a partir das necessidades da empresa Duas Rodas.

{\color{red}demostrar essas características em uma tabela comparativa}